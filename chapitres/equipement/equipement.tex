\section{\'Equipement}

\subsection{Armures}
Voici quelques armure que l'on peu trouver dans l'univers de \swfe. Il s'agit là plus d'une grille d'étalonnage que d'une liste exhaustive. La limite c'est l'imagination des joueurs et la volonté du MJ. Vous trouverez sur \citetitle{website:starwars-holonet} une quantité d'arme décrites dans les moindre détails.

\begin{dnditemtable}[ l c c c ]
    \textbf{Type} & \textbf{Armure} & \textbf{Poids} & \textbf{Prix} \\
    Veste en Cuir           & +1  & 10 & 50        \\
    Casque                  & +4  &  5 & 80        \\
    Armure Storm Trooper    & +6  & 20 & Militaire \\
    Armure mandalorienne    & +7  & 15 & 1000      \\
    M'uhk'gla (lourde)      & +14 & 30 & 2500      \\
    \hline
    \hline
    Bouclier                & +3  & 20 & 200       \\
    Bouclier Gungan         & +4  & 10 & 1500       
\end{dnditemtable}

\begin{center}
	\includegraphics[width=\linewidth]{img/equipement/mandalorian_heavy_armor.png}
	\caption{\emph{Armure lourde Mandalorienne}}
\end{center}

\subsection{Armes}

\subsubsection{Combat corps à corps}

\begin{dnditemtable}[ l c c c ]
    \textbf{Type} & \textbf{Dégats} & \textbf{Poids} & \textbf{Prix} \\
    Vibrolame				& For+d4    &  1 & 25		 \\
    Bâton           		& For+d4    &  1 & 10        \\
    Couteau laser			& 2d6+4     &  1 & 500		 \\
    Vibroépée				& For+d6+2  &  6 & 250		 \\
    Bâton électrique        & For+d8+2  & 10 & 700	
\end{dnditemtable}

\subsubsection{A distance}

La portée est donné en "case" si le jeux se fait sur table ou sur grille. Une case fait 2.5m, par ex. un Blaster de portée 10 a une portée réelle de 25m.

Entre parenthèse après les dégats on trouve la cadence de tir de l'arme. En nombre de projectile maximum envoyé par tour. Attention dés que le nombre de projectile envoyé dépasse 1, il y a un malus de -2 pour le recul.

\begin{dnditemtable}[ l c c c c ]
    \textbf{Type} & \textbf{Portée} & \textbf{Dégats} & \textbf{Poids} & \textbf{Prix} \\
    Blaster                 & 10  &  2d6+1 (1) & 4  & 200  \\
    Blaster Lourd           & 15  &  2d8   (1) & 4  & 400  \\
    Fusil Blaster           & 10  &  2d8   (3) & 6  & 800  \\
    Blaster de précision    & 200 &  2d10  (1) & 8  & 1200 \\
    Lance grenade           & 5   &  2d10  (1) & 7  & 500  \\
    Arbalète laser          & 8   &  2d6+1 (3) & 9  & 600  \\
    Canon blaster rotatif   & 20  &  2d8   (4) & 40 & 600
\end{dnditemtable}

\begin{description}[align=left]
    \item [Fusil Blaster DC-17m]
        Le blaster DC-17m peut être utilisé en trois modèles qui le rendent terriblement efficace pour des missions aussi diverses que celles accomplies par les commandos. Il peut tout d'abord être réglé en mode blaster conventionnel, il possède alors les caractéristiques d'un Fusil Blaster. Le deuxième mode de ce blaster est en sniper, il possède alors les caractéristiques d'un Blaster de précision. Enfin, il peut être utilisé en mode grenade, essentiellement pour faire des trous dans les parois blindées et très efficace pour déblayer une zone.

        Le changement de mode prend un tour complet s'il est effectué durant un combat. Cette arme coute 2000 Cr.

    \item [Arbalète Laser]
        Cette arme surtout utilisé par les Ewoks et les Wookies est capable d'envoyer jusqu'à 3 projectiles dans des directions différentes sur un même plan horizontal. C'est une arme très lourde qui demande un minimum de For d6+ pour être manipulé correctement.

    \item [Canon blaster rotatif]
        Les armes de ce gabarit nécessite une certaine force pour être utilisé (For d8+). De plus il est impossible de se déplacer lorsque l'on utilise cette arme.
\end{description}

\subsubsection{Armes Embarquées}
Ici on voit les armes montés sur les véhicules (militaire ou non). Ces armes peuvent être équipé ou changé à postériorie sur un véhicule, à condition que le MJ soit d'accord avec ça biensur.

\clearpage
\subsection{Armes Spéciales}

\subsubsection{Sabre Laser}
\label{sec:sabre-laser}

\begin{flushright}
	\vspace{-5\baselineskip}
	\includegraphics[width=5cm, angle=-25]{img/equipement/lightsaber01.png}
	\vspace{-1\baselineskip}
\end{flushright}

Le sabre laser est l'arme des chevaliers combattant avec la Force, qu'ils soient du Côté Obscur ou du Côté Lumineux. La lame est un faisceau d'énergie pure, produit par généralement trois cristaux polis contenus dans le manche. Seuls les utilisateurs de la Force ont la compétence nécessaire pour l'utiliser. Tout autre utilisateur aurait autant de chances de se blesser que de blesser ses adversaires. 

Il existe une multitude de cristaux à travers la Galaxie, mais seuls certains d'entres eux, assez rigides, ou bien parfaitement constitués, peuvent être utilisés, et c'est selon les types des cristaux, que la lame arborera une certaine teinte. Chaque cristal peut receler un pouvoir lorsqu'il est alimenté. A travers les âges, beaucoup de ces cristaux furent découverts mais ils sont très rares et difficiles à déceler. En voici quelques exemples : cristal de Solari, cristal Damind, cristal Kaiburr, \ldots

Mais les Jedi arrivèrent également à créer des cristaux industriellement pour alimenter leurs besoins. Ces cristaux peuvent être modifiés lors de leur croissance pour avoir une lame de la couleur voulue, mais toutes les couleurs ne sont pas disponibles. Ces cristaux sont toutefois moins maniables, mais plus puissants que les cristaux naturels qui sont, à l'inverse, plus maniables mais moins puissants. Ce sont, en général, les Sith et les Jedi Noirs qui utilisent ce type de cristaux car ils recherchent la puissance, et non la maniabilité. Historiquement, les Sith ont toujours préféré des lames de couleur rouge. 

Le Sabre laser est une arme très rare, chaque Jedi fabrique le sien et il est unique. Il n'existe doc que deux façons de s'en procurer un, le fabriquer ou le volé sur le cadavre d'un Jedi.

La fabrication d'un Sabre laser n'est pas aisé:
\begin{enumerate}
	\item Trouvé tous les composants, une cellule d'énergie, une lentille, un émetteur de lame et un cristal.
	\item Apprendre les rudiment du montage d'un sabre (Connaissance (Jedi) d6+).
	\item Maîtriser la Force suffisement pour assaiblé les différentes pièce (Maîtrise de la Force d6+).
\end{enumerate}

Une fois toutes les conditions remplies, le Jedi dont méditer pendant un mois pour assembler les pièces de son sabre. Quand le sabre est terminé, lors de sa première utilisation, le joueur fait un jet de Maîtrise de la Force, s'il fait 1, le sabre explose, le composant sont perdu et le héro prend un niveau de blessure. Il faudra tout recommencer.

\begin{dnditemtable}[ l c c c ]
    \textbf{Type} & \textbf{Dégats} & \textbf{Poids} & \textbf{Notes} \\
    Sabre Laser   & For+d6+8        & 2              & Parade +3, PA 4
\end{dnditemtable}

\textbf{PA (Pénétration d’armure)}: l’arme ou le projectile ignore les points d’armure indiqués. Une arme avec la capacité PA 4 ignore 4 points d’armure. Les points de PA supérieurs à la protection de la cible sont ignorés.

\begin{center}
	\vspace*{\fill}
	\includegraphics[width=0.7\linewidth]{img/equipement/jedi01.png}
	\vspace*{\fill}
\end{center}

\clearpage
\subsection{Véhicules}
\subsubsection{Véhicules Terrestres}
\begin{center}
    \vspace{-2\baselineskip}
    \includegraphics[width=0.9\linewidth]{img/equipement/landspeeder.png}
    \vspace{-1\baselineskip}
\end{center}
\begin{dnditemtable}[ l c c c c ]
    \textbf{Type} & \textbf{Acc/VMax} & \textbf{Rés.} & \textbf{Passager} & \textbf{Prix} \\
    Motojet         & 20/36           & 8(2)          & 1+1               & 6 000  \\
    Landspeeder     & 20/40           & 11(3)         & 1+3               & 12 000 \\
    Podracer        & 30/56           & 10(3)         & 1                 & 25 000 \\
    TB-TT\footnote{Transport Blindé Tout-Terrain}     & 5/16   & 110(100)  & 2+40    & Militaire 
\end{dnditemtable}

\subsubsection{Véhicules Aériens}
Les véhicules aérien sont pour la plus part man\oe{uvrable} en haute atmosphère mais ne permettent pas de voyage FTL\footnote{Faster Than Light}. 
\begin{center}
    \vspace{-1\baselineskip}
    \includegraphics[width=0.9\linewidth]{img/equipement/airspeeder-taxi.png}
    \vspace{-1\baselineskip}
\end{center}
\begin{dnditemtable}[ l c c c c ]
    \textbf{Type} & \textbf{Acc/VMax} & \textbf{Rés.} & \textbf{Passager} & \textbf{Prix} \\
    Air Speeder     & 20/40           & 16(4)         & 1+1               & 60 000
\end{dnditemtable}

\subsubsection{Vaisseaux Spaciaux}
Tous ces vaisseaux possèdent la navigation FTL.
\begin{dnditemtable}[ l c c c c ]
    \textbf{Type} & \textbf{Acc/VMax} & \textbf{Rés.} & \textbf{Passager} & \textbf{Prix}      \\
    Chasseur                & 50/700           & 20(5)         & 1+1               & 2M        \\
    Navette                 & 45/600           & 25(6)         & 2+5               & 5M        \\
    Transport               & 35/400           & 45(10)        & 2+20              & Militaire \\
    Croiseur                & 30/300           & 50(11)        & 15+985            & 200M      \\
    Batiment                & 25/200           & 55(13)        & 20+2980           & Militaire \\
    EDLM\footnote{\'Etoile de la mort} & 20/200           & 80(25)        & 100+5K            & Militaire 
\end{dnditemtable}
\vspace*{\fill}

\begin{center}
    \includegraphics[width=0.9\linewidth]{img/equipement/navette.png}
\end{center}

\begin{description}[align=left]
    \item [Chasseur léger]
        Beaucoup de vaisseaux rentre dans cette catégorie. Le X-Wing, le Tie, ...

    \item [Navette]
        Dans cette catégorie rentre les vaisseaux comme le Millenium Falcon ou le Slave 1.

    \item [Croiseur]
        Beaucoup plus rare dans les campagnes mais les MJ peuvent s'en servir.

    \item [Batiment]
        Il s'agit des Battleship, vaisseaux de grande taille, capable d'en contenir d'autre (type chasseur). Lourdement armé la plus part du temps. Le Destroyé Impériaux se classe dans cette catégorie.

    \item [Station Orbitale]
        Juste pour la forme car rarissime dans les campagnes de jeux. Les stations orbitales sont capable de détruire des planètes entieres.
\end{description}

\begin{center}
    \vspace*{\fill}
    \includegraphics[width=0.9\linewidth]{img/equipement/croiseur.png}
    \caption{Croiseur}
    \vspace*{\fill}
    \includegraphics[width=0.9\linewidth]{img/equipement/batiment.png}
    \caption{Batiment}
\end{center}