\subsection{Atouts}

Voici le liste des Atouts disponibles dans \swfe.

\subsubsection{Background}
\begin{description}[align=left]
    \item [Ambidextre]
    	\emph{[Novice, Agi d8+]}\\
        Votre héros utilise ses deux mains avec la même facilité.

    \item [Arcane (Force)]
    	\emph{[Novice, spécial, \^Ame d8+]}\\
        L'atout indispensable si votre héro doit utiliser la Force.

    \item [Brave]
    	\emph{[Novice, spécial]}\\
        Ceux qui possèdent cet Atout ont appris à maîtriser leurs peurs.

    \item [Chanceux]
    	\emph{[Novice]}\\
        Votre héros semble être béni par le destin.

    \item [Très Chanceux]
    	\emph{[Novice, Chanceux]}\\
        Votre héros semble être béni par le destin. Deux fois.

    \item [Costaud]
    	\emph{[Novice, Force et Vigueur d6+]}\\
        Votre héros est très corpulent ou tout simplement très athlétique.

    \item [Don des langues]
    	\emph{[Novice, Intellect d6+]}\\
        Le personnage a un don pour les langues.

    \item [Enragé]
    	\emph{[Novice]}\\
        Dès qu’il est blessé (ou même Secoué par une attaque physique) votre héros doit réussir un jet d’Intellect sans quoi il devient Enragé. Un personnage enragé attaque sans aucune retenue.

    \item [Guérison rapide]
    	\emph{[Novice, Vigueur d8+]}\\
        Votre héros récupère vite de ses blessures.

    \item [Noble]
    	\emph{[Novice]}\\
        Les personnages de haute naissance sont avantagés par la vie mais ont aussi plus de responsabilités.

    \item [Résistance à la Force]
    	\emph{[Novice, \^Ame d8+]}\\
        Votre héro, même s'il ne sait pas l'utiliser, est capable de résister à la Force quand elle est employé contre lui.

    \item [Grande résistance à la Force]
    	\emph{[Novice, Résistance à la Force]}\\
        Comme précédent mais avec un bonus de résistance double.

    \item [Riche]
    	\emph{[Novice]}\\
        Que votre héros soit né avec une cuillère en argent dans la bouche ou bien qu’il ait bien réussi en affaires, une chose est sûre : il est beaucoup plus fortuné que la plupart des gens.

    \item [Très riche]
    	\emph{[Novice, Riche ou Noble]}\\
        Votre héros est riche comme Crésus.

    \item [Séduisant]
    	\emph{[Novice, Vigueur d6+]}\\
        Votre héros a beaucoup de charme ou est très beau.

    \item [Très séduisant]
    	\emph{[Novice, Séduisant]}\\
        Votre héros est d’une beauté à couper le souffle.

    \item [Véloce]
    	\emph{[Novice, Agilité d6+]}\\
        Le héros se déplace très rapidement.

    \item [Vif]
    	\emph{[Novice, Agilité d8+]}\\
        Votre héros est né avec des réflexes presque surhumains.

    \item [Vigilant]
    	\emph{[Novice]}\\
        Peu de choses échappent à votre héros. Il est vigilant et très observateur.
\end{description}

\begin{paperbox}{Arcane (Force)}
    Remplace l'attout Arcanes. C'est l'atout qui représente la Force, l\^Ame étant assimilé au niveau de maîtrise de la Force.
\end{paperbox}

\begin{paperbox}{Résistance à la Force}
    Comme sont niveau supérieur Grande Résistance à la Force, cet atout remplace la résistance aux Arcanes. il correspond comme son nom l'indique à la capacité que le héro a de résisté à la force.

    Il confère au héro 2 points d’Armure contre les dégâts provoqués par la Force et ainsi qu'un bonus de +2 aux jets pour résister aux effets de ces pouvoirs. Utiliser un pouvoir bénéfique sur le personnage n'implique pas de malus, cet atout ne se déclenche qu'à la volonté du héro.

    Pour la grande résistance, le bonus d'Armure passe à 4.
\end{paperbox}

\subsubsection{Atouts de Combat}
\begin{description}[align=left]
    \item [Ambidextre]
    	\emph{[Novice, Agilité d8+]}\\
        Votre héros utilise ses deux mains avec la même facilité.
    \item [Ambidextre]
    	\emph{[Novice, Agilité d8+]}\\
        Votre héros utilise ses deux mains avec la même facilité.
    \item [Ambidextre]
    	\emph{[Novice, Agilité d8+]}\\
        Votre héros utilise ses deux mains avec la même facilité.
    \item [Ambidextre]
    	\emph{[Novice, Agilité d8+]}\\
        Votre héros utilise ses deux mains avec la même facilité.
    \item [Ambidextre]
    	\emph{[Novice, Agilité d8+]}\\
        Votre héros utilise ses deux mains avec la même facilité.
    \item [Ambidextre]
    	\emph{[Novice, Agilité d8+]}\\
        Votre héros utilise ses deux mains avec la même facilité.
    \item [Ambidextre]
    	\emph{[Novice, Agilité d8+]}\\
        Votre héros utilise ses deux mains avec la même facilité.
    \item [Ambidextre]
    	\emph{[Novice, Agilité d8+]}\\
        Votre héros utilise ses deux mains avec la même facilité.
    \item [Ambidextre]
    	\emph{[Novice, Agilité d8+]}\\
        Votre héros utilise ses deux mains avec la même facilité.
    \item [Ambidextre]
    	\emph{[Novice, Agilité d8+]}\\
        Votre héros utilise ses deux mains avec la même facilité.
\end{description}