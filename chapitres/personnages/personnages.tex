
\section{Création de Personnage}
\lettrine{L}{e} c\oe{}ur des jeux de rôle réside dans la possibilité de créer, améliorer et faire évoluer son propre personnage. Voilà comment ça fonctionne dans {\jedifont \doctitle}. 

\subsection{Les Attributs}
Chaque personnage commence le jeu avec d4 dans chaque Attribut et dispose de 5pt pour les améliorer. Améliorer l’un d’entre eux d’un type de dé (par exemple, de d4 à d6) coûte 1 point avec une limite : vous ne pouvez aller au-delà du d12.

\subsection{Compétences}
Les Compétences représentent les aptitudes apprises comme le Tir, le Combat, les connaissances professionnelles ou scientifiques et ainsi de suite. Elles sont générales et englobent tous les aspects qui leur sont reliés. Par exemple Tir englobe les fusils, les arcs, les lance-roquettes et toutes les armes à distance. Vous disposez de 15 points à répartir entre vos Compétences. Chaque dé de Compétence coûte 1 point (en commençant à d4) tant qu’il est inférieur ou égal à l’Attribut dont il dépend (noté entre parenthèses près du nom de la Compétence). Chaque dé de Compétence supérieur à l’Attribut dont il dépend coûte 2 points. De même que pour les Attributs, aucune Compétence ne peut dépasser d12.

\subsection{Atouts \& Handicaps}
Faire naître un héro digne de ce nom ne se limite pas à lui faire grimper ses attributs et ses compétences. Ce qui fait la singularité d’un héro et qui le rend fun à jouer c’est ses Atouts et ses Handicaps. De la blonde très séduisante au Vieux maître Jedi, vous prendrez toujours plus de plaisir à jouer un personnage unique dans ses détails que le jeune humain sans problème que vous êtes tout les jours.

Vous pouvez choisir jusqu’à 1 Handicap Majeur et 2 Handicaps Mineurs. Un Handicap Majeur donne 2 points et un Handicap Mineur 1 point.

Pour 2 points vous pouvez :
\begin{rebelist}
    \item Augmenter un Attribut d’un type de dé (vous pouvez le faire avant de prendre vos Compétences).
    \item Choisir un Atout.
\end{rebelist}

Pour 1 point vous pouvez :
\begin{rebelist}
    \item Gagner un point de Compétence.
    \item Doubler vos fonds de départ (si vous débutez avec 500\crg vous obtenez 500\crg de plus).
\end{rebelist}

\subsection{Races Jouables}
Vous pouvez choisir pour votre personnage n’importe quelle race disponible dans L’univers de Star Wars. L’orientation de chaque race est donnée à titre indicatif, chaque race possède des exceptions parmi ses héros.

\subsubsection{Humain}
\begin{samepage}
	\begin{flushright}
		\begin{tabular}{ l l }
			\textbf{Type} 			& Humanoïde \\
		   	\textbf{Planète} 		& Terre \\
		   	\textbf{Langage} 		& Basic \\
		   	\textbf{Orientation} 	& Neutre \\
		\end{tabular}
	\end{flushright}

	\vspace{-6\baselineskip}
	\includegraphics[width=5cm]{img/personnages/races/humain.png} 
\end{samepage}

Cette race comprend aussi bien les humains au sens strict (qu’ils soient originaires de Coruscant, de Correlia, de Kuat, de Naboo\ldots) que les humanoïdes dont les caractéristiques physiques, intellectuelles, sociales et culturelles sont suffisamment proches de celles des humains pour qu’il soit possible de les assimiler en termes de jeu.

\begin{description}[align=left]
\item [Adaptabilité] 	%CAP +2
	Les humains sont une race pleine de ressources, ils s’adaptent rapidement à toutes sorte de difficultés ou environnements.\\
	\emph{Compétence à d6}
\end{description}
\subsubsection{Barabel}
\begin{samepage}
	\begin{flushright}
		\begin{tabular}{ l l }
			\textbf{Type} 			& Reptile \\
		   	\textbf{Planète} 		& Barab I \\
		   	\textbf{Langage} 		& Barabel \\
		   	\textbf{Orientation} 	& Obscur \\
		\end{tabular}
	\end{flushright}

	\vspace{-5\baselineskip}
	\includegraphics[width=5cm]{img/personnages/races/barabel.png}
\end{samepage}

Originaire de Barab I, les Barabels sont restés une race relativement primitive et isolée. Les Barabels vivent en clans dans un société principalement matriarcale. Ils sont fascinés par la guerre, la violence et les armes. Les Barabels ne sont pas profondément cruels, mais ils restent agressifs de nature. En raison des nombreux rituels précédent les négociations, la diplomatie avec les Barabels est un exercice compliqué.

Le Barabel adulte est un reptile bipède dont la taille dépasse toujours les deux mètres. Sa dentition est formée d’une multitude de dents en forme d’aiguilles qui peuvent atteindre jusqu’à cinq centimètres de long et ses mains sont équipées de griffes puissantes.

\begin{description}[align=left]
\item [Enfance difficile] 	% CAP +1 +1
		De par l’environnement hostile de leur planète natale, les Barabels possèdent une résistance accrue à la chaleur et aux radiations.\\
		\textit{+4 en Résistance à la chaleur}\\
		\textit{+4 en Résistance aux radiations}
\item [\OE{il} Ophidien] 	% CAP +1
		Les yeux ophidiens du Barabel lui permettent de capter la plupart des ondes lumineuses allant du jaune à l’infrarouge, mais il confond facilement les couleurs tirant dans les bleus et violets.\\
		\textit{Infravision}
\item [Arme naturelle]		% CAP +1
		Les mains des Barabels sont équipés de puissante griffes.\\
		\textit{For + d6 de dégâts}
\item [Balayage]			% CAP +2
		Les Barabels utilise leur appendice caudal d’instinct dans les combats.\\
		\textit{+ Atout Balayage}
\item [Primitif]			% CAP -3
		Les Barabels sont une race encore primitive.\\
		\textit{Int <= d6}
\item [Dur d’oreille]		% CAP -1
		Les Barabels en tant que reptilien ne possède pas d’oreille, ils entendent par vibrations.\\
		\textit{Dur d’oreille (Mineur)}
\end{description}
\subsubsection{Bothan}

\begin{samepage}
	\begin{flushright}
		\begin{tabular}{ l l }
			\textbf{Type} 			& Félin \\
		   	\textbf{Planète} 		& Bothawui \\
		   	\textbf{Langage} 		& Bothese \\
		   	\textbf{Orientation} 	& Lumineux \\
		\end{tabular}
	\end{flushright}

	\vspace{-5\baselineskip}
	\includegraphics[width=5cm]{img/personnages/races/bothan.png}
\end{samepage}

Les Bothans sont des humanoïdes trapus, dont le corps est recouvert d’une épaisse fourrure pouvant varier du blanc-cassé à brun très foncé.
Le peuple bothan est originaire de la planète Bothawui, un monde cosmopolite épargné des troubles de la Guerre Civile Galactique en raison de la neutralité officielle du gouvernement bothan. Plusieurs colonies ont également été construites sur des planètes proches, telles que Kothlis, qui est désormais le siège d’une importante communauté. Toutes ces colonies forment l’Espace Bothan.\\
La structure sociale des Bothans est constituée de clans familiaux, dont le nom est inclus dans le nom de chaque Bothan, à la suite d’une apostrophe.

\begin{description}[align=left]
\item [Agilité du Félin] 			% CAP +2
		Les Bothans possède la grâce de leurs ancêtres félins.\\
		\textit{Commence avec d6 Agi}

\item [Service de renseignement] 	% CAP +1
		Depuis plus de 300 ans ces êtres intelligents et rusés ont perfectionné leurs façons de faire, et ont développé un vaste réseau d’espions et d’informateurs destiné à recueillir toutes sortes d’informations sur les sujets les plus importants.\\
		\textit{d6 en Réseaux}

\item [Comme en plein jour] 		% CAP +1
		Grâce à leurs yeux de félins, les Bothans voient parfaitement dans l’obscurité.\\
		\textit{Vision Nocturne}

\item [Déplacement rapide] 			% CAP +2
		Les Bothans, s’ils se baissent sur leur quatre pattes peuvent atteindre des vitesses de 80 km/h.\\
		\textit{All = 10}

\item [Frêle] 						% CAP -2
		De constitution moins résistante, les Bothans sont moins adaptés aux combats rapprochés.\\
		\textit{-1 Résistance}

\item [Mauvaise réputation] 		% CAP -1
		Les ruses conduites par ce peuple, ainsi que l’opacité inhérente au Réseau Bothan, ne jouent pas en leur faveur. Certains leur reprochent de jouer sur les deux tableaux et de vendre des informations aussi bien à l’Alliance qu’à L’Empire.\\
		\textit{\'Etranger}

\item [Prudent] 					% CAP -1
		Les Bothans ne font rien à la légère, ils ne laissent nulle place au hasard et chaque décision est mûrement réfléchie. Ils ne connaissent pas l’urgence.\\
		\textit{Prudent}
\end{description}
\subsubsection{Chiss}
\begin{samepage}
	\includegraphics[width=5cm]{img/personnages/races/chiss.png}

	\vspace{-9\baselineskip}

	\begin{flushright}
		\begin{tabular}{ l l }
			\textbf{Type} 			& Humanoïde \\
		   	\textbf{Planète} 		& Csilla \\
		   	\textbf{Language} 		& Cheunh \\
		   	\textbf{Orientation} 	& Obscur \\
		\end{tabular}
	\end{flushright}
	\vspace{4\baselineskip}
\end{samepage}

Les Chiss font en moyenne 1,80m et ont une morphologie humaine. Cependant, il est impossible de les confondre à cause de leur peau bleue et de leurs yeux d’un rouge éclatant. Ils ont toujours des cheveux noirs, bien qu’avec les années, certains voient des cheveux blancs apparaitre. \\ 

La société Chiss est très évoluée. Ils ont de l’intérêt pour les arts et la science et maintiennent une puissante force militaire. Ils ont la réputation d’être de fins stratèges militaires mais leur façon de penser se retrouve dans tous les domaines de la vie quotidienne. Ils réfléchissent et pensent à différents points de vue et aux alternatives lorsqu’ils doivent prendre une décision.  

\begin{description}[align=left]
\item [Charismatique] 			% CAP +2 +2 +1
		Les Chiss sont des êtres charismatique habitué à commander des armées.\\
		\emph{+2 Cha}\\
		\emph{Commandement}\\
		\emph{d6 Connaissance (Combats)}
\item [Aquité visuelle] 		% CAP +1
		Les modifications qu’ont subies leurs yeux leur ont également donné une plus grande acuité visuelle.\\
		\emph{d6 Perception}
\item [Arrogant] 				% CAP -2
		Les Chiss sont fréquemment perçus par le reste de la galaxie, comme un peuple arrogant, calculateur et distant.\\
		\emph{Arrogant}
\item [Insensible à la Force] 		% CAP -1
		Les Chiss ne sont pas connus pour être une espèce sensible à la Force. Ils n’ont eut qu’un seul exemple d’individu sensible, en la personne de Sev’rance Tann. Cette dernière avait optée pour le côté obscur.\\
		\emph{A la création, l'augmentation de l'\^Ame coute 2pt}
\end{description}
\subsubsection{Drall} 
	\vspace{-1\baselineskip}
	\begin{flushright}	
		\begin{tabular}{|l|r|}
		\textbf{Type} 		 & Félin \\
		\textbf{Planete} 	 & Drall \\
		\textbf{Langage} 	 & Drallish \\
		\textbf{Orientation} & Neutre\\
		\end{tabular}
	\end{flushright}
	\vspace{-6\baselineskip}
	\includegraphics[width=6cm]{img/personnages/races/drall.png}

Les Dralls sont de petits mammifères (90~cm à 1,50~m), recouverts d’une fourrure brune et noire (gris pour les plus âgées), ils ont un museau et de grand yeux noirs. Les mains et les pieds se terminant par de petites griffes acérées. Les femelles sont généralement plus grandes et plus fortes. Les Drall ne portent pas de vêtements (comme toute créature à fourrure) et s’ornent de beaucoup de bijoux, ils cherchent constamment de nouvelles richesses quitte à se mettre hors la loi. Très avides de connaissances ils aiment apprendre juste pour le plaisir d’en savoir toujours plus.\\

\begin{description}[align=left]
\item[Soif de connaissance]    % +1
 	En tant qu’espèce, les Dralls sont surtout des êtres qui privilégient l’utilisation de leur cerveau à celle de leurs muscles. Ce sont des chercheurs à l’esprit méthodique, des observateurs studieux, et ils se considèrent eux-mêmes comme les meilleurs théoriciens de la galaxie. \\
 	\textit{d6 Intellect}

\item[Arsène Félin]    % +2
 	Le goût des Dralls pour les bijoux sans en avoir les moyens les ont poussés à devenir expert dans l’art de subtiliser tout ce qui brille. Ils ont donc un bonus naturel de +2 en Escalade, Crochetage et Discrétion \\
 	\textit{Atout Voleur}
 	
\item[Griffes rétractables]    % +1
	Leurs mains et leurs pieds sont terminés par de petites griffes très acérées.\\
	\textit{Arme naturelle (For+d6)}\\
    
\item[Vision Nocturne]  % +1
	Dans l’obscurité, les yeux des Dralls amplifient la lumière et ignorent les malus aux attaques pour des obscurités Légères ou Forte.\\
	\textit{Vision nocturne}\\

\item[Mourir moins bête]       % -2
 	Le personnage veut tout savoir. Dès qu’un Drall découvre quelque chose de nouveau, il ne peut pas s’empêcher de s’y intéresser.\\
    \textit{Handicap Curieux} \\

\item[Frêle]       				% -2
 	Les Dralls sont des intellectuels, agiles de par leur nature, mais ils ne sont pas taillés pour le combat. Ils ont tendance à faire usage de la ruse pour esquiver les affrontements physiques.\\
    \textit{Handicap Frêle (-1 Résistance)}
\end{description}


\subsubsection{Droïde}
\begin{samepage}
	\vspace{4\baselineskip}
	\begin{tabular}{ l l }
		\textbf{Type} 			& Artificiel \\
	   	\textbf{Planète} 		& Multiple \\
	   	\textbf{Langage} 		& Binaire \\
	   	\textbf{Orientation} 	& Neutre \\
	\end{tabular}

	\vspace{-11\baselineskip}

	\begin{flushright}
		\includegraphics[width=6cm]{img/personnages/races/droide.png}
	\end{flushright}
	\vspace{-2\baselineskip}
\end{samepage}

Les Droïdes ne sont pas une race à proprement parlé mais des entités artificielles créés par d’autres races. Il peuvent être de Combat, de Protocole, de Compagnie, \ldots 
De par leur nature artificielle les Droïdes ne peuvent et ne pourront jamais utiliser la Force, c’est une notion qui leur est totalement étrangère.

\begin{description}[align=left]
\item [Créature artificielle] 	% CAP +2
		Les droïdes ne ressentent pas la douleur de blessures ou de la perte d’un membre.\\
		\emph{+2 pour se remettre d’un état secoué}\\
		\emph{Pas de bonus aux attaques ciblées}\\
		\emph{Pas de malus de blessure}

\item [Immunisé] 				% CAP +1
		Les maladies et les poisons sont sans effet sur les droïdes.\\
		\emph{Immunisé}

\item [Ambidextre] 				% CAP +2
		Un droïde ne fait pas de différence entre un membre et un autre, il peut utiliser n’importe lequel indistinctement.\\
		\emph{Ambidextre}

\item [Manque pas d’air] 		% CAP +2
		Les droïdes n’ont pas besoin de respirer, il peuvent stationné dans des lieux dépourvu d’atmosphère. Ils restent cependant sensible à la température.\\
		\emph{Ambidextre}

\item [Pas d’\^Ame] 			% CAP -3
		Il est impossible pour un droïde d’utiliser la force.\\
		\emph{\^Ame <= d6}
		\emph{Compétence Arcane (Force) interdite}

\item [Outsider] 				% CAP -2
		Le droïdes sont considéré comme des servants par les autres espèces, ils n’ont pas de droits et ne sont pas considéré comme faisant parti de la société.\\
		\emph{\'Etrangé}
\end{description}
\subsubsection{Gungan}
\begin{samepage}
	\vspace{-1\baselineskip}
	\includegraphics[width=6cm]{img/personnages/races/gungan.png}
	\vspace{-12\baselineskip}

	\begin{flushright}
		\begin{tabular}{ l l }
			\textbf{Type} 			& Amphibien \\
		   	\textbf{Planète} 		& Naboo \\
		   	\textbf{Langage} 		& Gunganese \\
		   	\textbf{Orientation} 	& Lumineux \\
		\end{tabular}
	\end{flushright}

	\vspace{7\baselineskip}
\end{samepage}

Natifs de la planète Naboo, les Gungans vivent dans des cités sous-marines. La physiologie d’un Gungan est de type humanoïde, quoique plus grand et plus fin qu’un Humain. Les Gungans possèdent de longues oreilles tombantes, des narines souples et des membranes rétractiles protégeant les yeux lors de leur déplacement aquatique. Leurs articulations sont libres et les ligaments très souples, permettant aux amphibiens de nager avec aisance sous l’eau.

\begin{description}[align=left]
\item [Aquatique] 					% CAP +2 +1
		Les gungans vivent dans les grandes étendues d’eau de Naboo, ils ne peuvent se noyer. Ils se déplacent sous l’eau beaucoup plus vite que n’importe qu’elle autre espèce.\\
		\emph{d6 Natation}\\
		\emph{Allure sous l’eau = d Natation}
		
\item [Mollusque] 					% CAP +3
		Leurs articulations sont libres et les ligaments très souples, permettant aux amphibiens de nager avec aisance sous l’eau. Cette souplesse leur permet de se faufiler dans des endroits étroits et d’esquiver les attaques avec plus de réussite.\\
		\emph{Atout Esquive}

\item [Craint la chaleur] 			% CAP -2
		Leur physiologie aquatique rend les Gungans plus sensible aux fortes températures et à la sécheresse d’un climat.\\
		\emph{-4 pour résister à la chaleur}

\item [Langue bien pendue] 			% CAP -1
		Les Gungans passent le plus clair de leur temps à parler, de tout et n’importe quoi, surtout n’importe quoi. Néanmoins il arrive que la parole dépasse leur pensées et qu’ils livrent des secrets qu’ils n’auraient pas du livrer.\\
		\emph{Handicap Bavard}

\item [Maladroit] 					% CAP -1
		La culture Gungan est une relation poussée entre la nature et l’individu. Ils essaient de ne pas utiliser la technologie, se servant de ce que la Nature propose. Ils n’ont que peu l’habitude de la technologie conventionnelle et évitent de la manipuler sous peine de détraquer tout ce qu’ils touchent.\\
		\emph{Handicap Deux mains gauches}
\end{description}
\subsubsection{Miraluka}
\begin{samepage}
	\vspace{-2\baselineskip}
	\includegraphics[width=4.5cm]{img/personnages/races/miraluka.png}

	\vspace{-5\baselineskip}

	\begin{flushright}
		\begin{tabular}{ l l }
			\textbf{Type} 			& Humanoïde \\
		   	\textbf{Planète} 		& Alpheridies \\
		   	\textbf{Langage} 		& Miralukese \\
		   	\textbf{Orientation} 	& Lumineux \\
		\end{tabular}
	\end{flushright}
\end{samepage}

Les Miraluka ressemblent beaucoup aux Humains mis à part le fait que leurs cavités oculaires sont vides et qu’ils sont capables de voir à travers la Force. \'A l’origine, les pacifiques Miraluka vivaient sur un monde dont le nom n’est pas passé à la postérité et qui entra dans une phase d’instabilité géophysique et géo-chimique durant laquelle l’atmosphère de la planète commença à s’évacuer dans l’espace. Des éclaireurs miraluka partirent à la recherche d’un monde où pourrait s’installer leur peuple et trouvèrent une planète habitable dans le système d’Abron : Alpheridies. 

Les Miraluka ont également tendance à masquer le haut de leur visage avec des viseurs ou des morceaux de tissus pour dissimuler leurs orbites vides, en particulier lorsqu’ils voyagent, pour ne pas attirer l’attention.


\begin{description}[align=left]
\item [Sensibilité raciale à la force] 	% CAP +2
		Il est extrêmement rare qu’une espèce toute entière soit sensible aux flux de la Force et il n’est guère étonnant que certains Miraluka, parmi ceux qui maîtrisaient le mieux la Force, aient rejoint l’Ordre Jedi.\\
		\textit{d6 \^Ame}

\item [Vision de force] 			% CAP +2 +2
		Ils perçoivent leur environnement grâce à la Force. Cette «~vision~» est si puissante que s’ils «~regardent~» un Jedi ou un Sith à travers elle, ils verront les radiations de Force qu’ils dégagent. Il faut néanmoins noter que la connexion à la Force varie en fonction de chaque Miraluka.\\
		\textit{d6 compétence Force}\\
		\textit{Pouvoir (Vision de Force) permanent et gratuit}

\item [Aveugle dans la lumière] 	% CAP -2 -2
		Du fait qu’ils n’aient pas de globe oculaire, les Miraluka voient par la force, tout ce qui n’est pas lié à la Force leur échappe. Les droïdes, par exemple, étant des «~vides de Force~» ne leur apparaissent pas. Il leur est impossible de lire.\\
		\textit{Handicap (Aveugle)}\\
		\textit{-1 Parade}
\end{description}
\subsubsection{Togruta}

\begin{tabular}{ l l }
	\textbf{Type} 			& Humanoïde \\
   	\textbf{Planète} 		& Shili \\
   	\textbf{Language} 		& Togru \\
   	\textbf{Orientation} 	& Lumineux \\
\end{tabular}

\vspace{-9\baselineskip}
\begin{flushright}
\includegraphics[width=7cm]{img/races/togruta.png}
\end{flushright}

\vspace{-2\baselineskip}

Les Togrutas sont des humanoïdes dont l'apparence induit souvent en erreur les observateurs peu attentifs et les fait passer pour des Twi'leks. Ils possèdent une couleur de peau rouge vif, de la même manière là encore que certaines sous-espèces de Twi-leks. Leurs yeux sont entourés d'un grand cercle blanc, et leurs montrals sont également de cette couleur. Enfin leurs lekkus sont bariolés, de manière à assurer un camouflage efficace. Le Togruta adulte atteint en général une taille moyenne de 180 cm. 


\begin{description}[align=left]
\item [Agilité] 				% CAP +2
		Les Togrutas sont des être particulièrement agile.\\
		\emph{d6 \^Agi}
\item [Montrals] 				% CAP +1 +2
		Les montrals des Togrutas renferment de puissants organes sensoriels qui réagissent aux ultrasons. En pratique cela leur permet de se repérer dans leur environnement de manière efficace et autonome, ainsi que de détecter la présence d'éventuels prédateurs sur leur monde natal, Shili..\\
		\emph{d6 Perception}\\
		\emph{Sixieme Sens}
\item [Predateur né] 			% CAP +1
		Les Togrutas sur leur planète d'origine sont des prédateurs habitué à chasser. Ils aiment leur proies fraiche.\\
		\emph{d6 Discrétion}
\item [Mauvaise réputation] 	% CAP -1
		Leur réputation n'est pas toujours excellente car des rumeurs anciennes prétendent que les Togrutas sont capables d'injecter un poison mortel à leur victime, ce qui incite évidemment à la méfiance.\\
		\emph{\'Etranger}
\item [Frèle] 					% CAP -2
		Les Togrutas sont de corpulence élancé et sont moins résistant que d'autres races.\\
		\emph{-1 Résistance}
\item [Un pour tous] 			% CAP -1
		Les Togrutas sont une espèce très loyale envers leur compagnons d'aventure, en laisser un dans l'embarras alors qu'il y avait une chance même minime de le sauver est impensable.\\
		\emph{Loyal}
\end{description}
\subsubsection{Twi'Lek}
\begin{samepage}
	\vspace{-1\baselineskip}
	\includegraphics[width=6cm]{img/races/twilek.png}
	\vspace{-5\baselineskip}
	\begin{flushright}
		\begin{tabular}{ l l }
			\textbf{Type} 			& Humanoïde \\
		   	\textbf{Planète} 		& Ryloth \\
		   	\textbf{Language} 		& Ryl \\
		   	\textbf{Orientation} 	& Neutre \\
		\end{tabular}
	\end{flushright}
\end{samepage}

Les Twi'leks sont de grands humanoïdes, dont la peau très pigmentée peut avoir différentes couleurs selon les individus : rouge, jaune ou encore bleue par exemple. Leur trait le plus caractéristique est la paire de tentacules, appelés "lekkus", qui prend sa base au sommet de leur crâne.

Les Twi'leks utilisent leurs lekkus quand ils parlent leur langage d'origine, le twi'leki. Il s'agit d'un langage combinant communication orale et gestes, les mots étant accompagnés et complétés par les mouvements des lekkus.

La société twi'lek est divisée en deux castes très distinctes : les marchands et les guerriers.

% Les Twi'leks sont une race spéciale puisque c'est la seule à proposer des capacités différente pour les mâles et pour les femelles de la race. Attention toutefois à ce que les deux restent équilibré et à ne pas dépasser les +2 de capacité sans quoi les mâles et les femelles seraient trop différents.
\begin{description}[align=left]
\item [Rusé \& Astucieux] 			% CAP +2
		Les Twi'leks n'ont jamais eu la vie facile, entre leur monde natal pas vraiment amical et les hordes de criminels qui en veulent à leur Ryll, il leur a fallu faire preuve d'astuce pour composer avec tout cela.\\
		\emph{d6 Int}
\item [Ni chaud ni froid] 			% CAP +2
		Ryloth la terre natale des Twi'leks est composé de deux faces, l'une en permanence au soleil, à près de 300° et l'autre en permance dans l'obscurité. Le tout parcouru par de violentes tempêtes. Des conditions qui font des natifs des êtres particulièrement résistants à leur environnement.\\
		\emph{+4 pour résister aux effets négatifs de l’environnement}
\item [Lekkus Speaking] 			% Gratuit
		En plus du Ryl, les Twi'leks, grâce à leur lekkus sont capable de parler le twi'leki, ce qui s'avère difficile pour les autres races.\\
		\emph{Connaissance (twi'leki)}
\item [Belle plante (Femelles)] 	% CAP +2
		Toutes les femelles Twi'lek sont belle, au point que leur propre mâle les vendent aux pirates et autres criminels de passage pour arrondir les fins de mois.\\
		\emph{Séduisant (Cette capacité ne s'applique qu'aux personnages de sexe féminin)}
\item [Immunisé (Mâles)] 			% CAP +2
		Physiologiquement, les Twi'leks sont capables de résister à certaines toxines et maladies.\\
		\emph{Guérison rapide (Cette capacité ne s'applique qu'aux personnages de sexe masculin)}
\item [Frêles] 						% CAP -2
		Les Twi'leks sont de constitution moins résistance que les autres races.\\
		\emph{-1 Résistance}
\item [Prudent] 					% CAP -1
		Les Twi'leks en êtres intelligent prennent le temps de la réflexion et ne font pas les choses sans y réfléchir avant.\\
		\emph{Prudent}
\item [Hutt(er)] 					% CAP -1
		Les problèmes liés au commerce du Ryll ont obligé les Twi'lek à vendre leurs femmes pour présever leur monde des criminels. Les Hutt sont les premiers client de ce traffic et les Twi'leks libre ont du mal à garder leur calme face à un Hutt.\\
		\emph{Ennemi Racial (Hutt)}
\end{description}
\subsubsection{Wookie}
\begin{samepage}
	\begin{flushright}
		\begin{tabular}{ l l }
			\textbf{Type} 			& Bipèdes \\
		   	\textbf{Planète} 		& Kashyyyk \\
		   	\textbf{Langage} 		& Shyriiwook \\
		   	\textbf{Orientation} 	& Lumineux \\
		\end{tabular}
	\end{flushright}
	\vspace{-6\baselineskip}
	\includegraphics[width=6cm]{img/personnages/races/wookie.png}
\end{samepage}

Les Wookies sont de grands bipèdes à fourrure dépassant couramment les deux mètres de haut. Ils sont originaires de la planète Kashyyyk et n’ont que très peu de communautés en dehors de leur monde natal. Capables de vivre plusieurs siècles, les Wookies sont également dotés de longues griffes rétractiles, qu’ils utilisent principalement pour s’accrocher à la végétation dense de Kashyyyk. Leur honneur leur interdit formellement d’utiliser ces griffes comme armes lors d’un combat.

\begin{description}[align=left]
	\item [Force de la nature] 				% CAP +2
		L’imagerie populaire veut que les Wookies soient physiquement la race la plus forte de la galaxie (en tous cas, par rapport à sa taille).\\
		\textit{d6 For}

	\item [Increvable] 						% CAP +2 +3
		Les Wookies possèdent, entres autres, de remarquables capacités de récupération, et sont capables de survivre à des blessures très graves.\\
		\textit{Atout (Increvable)}\\
		\textit{Atout (Combatif)}

	\item [Shyriiwook] 						% CAP -1
		Les Wookies parlent entre eux leur langage, le Shyriiwook, un dialecte très complexe, en raison du mélange de feulements, rugissements, gestes et autres bruits nécessaires à son usage. Mais leur cordes vocales ne leur permettent pas de parler le Basic comme toutes les autres espèces mais ils le comprennent. La plus part des individus qui fréquentent régulièrement les Wookies comprennent le Shyriiwook dans les grandes lignes.\\
		\textit{Ne parle pas le Basic}

	\item [Force \& Honneur] 				% CAP -2
		Comme de nombreux peuples mettant en avant des valeurs comme l’honneur, les Wookies pratiquent les serments et la "dette de vie". Celle-ci peut les amener à défendre jusqu’à la mort un étranger (même d’une autre race) auquel ils pensent devoir une grande faveur. Une dette de vie est définitive et rien ne peut la lever.\\
		\textit{Code d’honneur}

	\item [Ennemis jurés] 					% CAP -1
		Les ennemis jurés des Wookies, les Trandoshans, se firent à une époque un malin plaisir à chasser et à capturer les wookies.\\
		\textit{Ennemi Racial (Trandoshans) -4 Cha}

	\item [Il faut partir à point] 			% CAP -1
		De par leur stature, les Wookies ne sont pas les êtres les plus vif de la galaxie.\\
		\textit{Allure 5}
\end{description}
\subsubsection{Zabrak}
\begin{samepage}
	\begin{flushright}
		\begin{tabular}{ l l }
			\textbf{Type} 			& Humanoïde \\
		   	\textbf{Planète} 		& Iridonia \\
		   	\textbf{Langage} 		& Zabraki \\
		   	\textbf{Orientation} 	& Obscur \\
		\end{tabular}
	\end{flushright}
	\vspace{-6\baselineskip}
	\includegraphics[width=5cm]{img/personnages/races/zabrak.png}
\end{samepage}

Originaires de la planète Iridonia, les Zabraks sont des humanoïdes d’une taille pouvant aller de 1,60 mètre à 2,10 mètres, et dont la tête était recouverte de petites cornes et le corps de tatouages, qui donnent à cette espèce un aspect parfois effrayant. Très tôt dans l’histoire galactique les Zabraks ont atteint un niveau de technologie élevé et ont pu coloniser plusieurs mondes extérieurs aux alentours de leur planète natale. On estime que cette espèce a ainsi établi huit colonies dans la Bordure Extérieure, et que les différents groupes coloniaux ont pu prospérer de façon suffisamment importante pour que les Zabraks s’identifient eux-mêmes selon les colonies d’où ils viennent, plutôt que par rapport à Iridonia uniquement.

\begin{description}[align=left]
	\item [Endurance] 				% CAP +2
		Très tôt dans l’histoire galactique les Zabraks ont atteint un niveau de technologie élevé et ont pu coloniser plusieurs mondes extérieurs aux alentours de leur planète natale. Cette vague de colonisation a forcé l’espèce à s’adapter et à se renforcer.\\
		\textit{d6 Vig}

	\item [Braveheart] 				% CAP +2 +1
		Les Zabraks sont des explorateurs courageux et des guerriers que rien n’effraie.\\
		\textit{Atout (Brave)}\\
		\textit{d6 Combat}

	\item [Survivor] 				% CAP +1
		Les Zabraks présentent un instinct de survie supérieur à la plupart des autres espèces.\\
		\textit{d6 Survie}

	\item [Fierté mal placé] 				% CAP -2 -2
		Les Zabraks dans leur ensemble possèdent un fort caractère et une volonté de fer.\\
		\textit{Handicap (Présomptueux)}
		\textit{Handicap (Arrogant)}
\end{description}

%\subsection{Races Non Jouables}
%\input{chapitres/personnages/races/hutt}
%\input{chapitres/personnages/races/jawa}

\clearpage
\subsection{Compétences}
Vous trouverez ici les Compétences utilisables dans \swfe. L’utilisation d’une Compétence en situation normale ne devrait pas donner lieu à un jet de dé. Seules les situations de stress, lorsque le temps est compté ou que la réussite est loin d’être acquise, devraient donner lieu à un test de Compétence.

On trouve entre parenthèses l’attribut associé à la compétence.

\begin{description}[align=left]
    \item [Combat (Agi)]
        Combat englobe toutes les attaques de corps à corps quelle que soit l’arme de mêlée utilisée.

    \item [Conduite (Agi)]
        Conduite permet à votre héros de conduire les véhicules terrestres ou aéroglisseurs courants de l’univers Star Wars.

    \item [Connaissance (Int)]
        Connaissance est une Compétence passe-partout qu’il faut spécialiser. Elle inclue aussi les langues autre que la langue natale et le Basic. Par exemple on peut utiliser les spécialisations suivante : Force, Jedi, Sith, Aéronautique, Informatique, Bataille, \ldots Dans tous les cas, on parle de connaissances théoriques et non pratiques.

    \item [Discrétion (Agi)]
        Discrétion représente la faculté à se cacher et à se déplacer en silence, mais aussi de camoufler des objets ou de subtiliser de petits objets à l’insu de tous.

    \item [\'Equitation (Agi)]
        \'Equitation permet de monter, contrôler et chevaucher tout animal domestiqué.

    \item [Escalade (For)]
        Les personnages peuvent être amenés à escalader des obstacles ou grimper une falaise pour prendre l’avantage du terrain lors d’une attaque, ou encore pour échapper à un ennemi trop coriace.

    \item [Intimidation (\^Ame)]
        Intimidation est l’art d’effrayer un adversaire par la force de sa volonté, que ce soit par une menace ouverte ou voilée ou tout simplement avec un énorme flingue.

    \item [Jeu (Int)]
        Voici un moyen rapide pour simuler une demi-heure d’une partie de jeu sans lancer les dés pour la moindre phase du jeu en question.

    \item [Lancer (Agi)]
        Lancer s’applique à toutes les armes qui se lancent, grenades, couteaux, haches, lances, etc.

    \item [Natation (Agi)]
        Natation détermine si un personnage nage ou coule comme une pierre lorsqu’il se trouve dans l’eau, ainsi que sa vitesse de déplacement en milieu aquatique.

    \item [Navigation (Int)]
        Navigation est la capacité du personnage à voyager et à se repérer dans l’espace. Il englobe aussi l’entretien journalier de l’équipement utilisé.

    \item [Maîtrise de la Force (\^Ame)]
        C’est la compétence qui mesure le niveau de votre héro à l’utilisation de la Force. Cette compétence ne sert à rien sans l’Atout Arcane (Force).

    \item [Perception (Int)]
        Perception représente la vigilance d’un héros et son habilité à découvrir objets ou indices.

    \item [Pilotage (Agi)]
        Pilotage permet d’utiliser tous les types d’appareils aériens ou spatiaux communs.

    \item [Piratage (Int)]
        Cette compétence permet à celui qui la possède de pirater les systèmes électroniques comme les serrures, ou les systèmes de contrôle dans un vaisseau ou un base.

    \item [Pistage (Int)]
        Pistage permet de suivre les traces d’un ou de plusieurs individus sur tout type de terrain.

    \item [Recherche (Int)]
        Recherche permet d’obtenir des informations dans une bibliothèque, sur l’HoloNet, dans les journaux ou toute autre source écrite.

    \item [Réparation (Int)]
        Réparation représente la capacité à remettre en état gadgets, véhicules, armes et autres machines.

    \item [Réseaux (Int)]
        Réseaux permet d’obtenir des informations dans la rue, les bars ou par des contacts en utilisant la menace, la corruption ou en offrant des verres. Cette compétence peut aussi représenté les informateurs du héro.

    \item [Sarcasme (Int)]
        Sarcasme est une attaque contre l’amour-propre d’un individu en le ridiculisant par la parole ou le geste.

    \item [Soins (Int)]
        Soins consiste à savoir comment guérir les plaies et traiter les blessures.

    \item [Survie (Int)]
        Survie permet de trouver nourriture, eau ou abri en milieu hostile.

    \item [Tir (Agi)]
        Tir concerne toute tentative pour toucher une cible avec n’importe quelle arme à distance (arc, pistolet, lance-roquettes, etc\ldots).
\end{description}

\begin{paperbox}{Piratage}
    Cette compétence vient remplacer le Crochetage qui dans Star Wars n’a pas une grosse utilité. La compétence Piratage (Informatique) va s’utiliser de la même façon mais sur l’Intellect au lieu de de l’Agilité. L’informatique pourra être utilisé dés qu’un ordinateur entre en scène. On pourra par exemple déverrouiller des sas de vaisseau ou stoppé un signal d’alarme ou encore désactiver des capteurs à plus haut niveau.

    Les Droïdes n’en sont pas systématiquement pourvu, cela dépend de leur programmation.
\end{paperbox}

\begin{paperbox}{Navigation}
    La navigation existe mais son domaine d’application change un peu, on ne parle pas dans Star Wars de bateau mais de vaisseaux spatiaux. La navigation est alors la capacité du héro à s’orienter dans le vide sidéral. Mais cela s’applique aussi au sol pour se repérer sur une carte par exemple.
\end{paperbox}

\vspace*{\fill}
\hspace*{-0.5\columnsep}
\includegraphics[width=\textwidth]{img/personnages/skills-01.png}

\clearpage
\subsection{Handicaps}

Les Handicaps\footnotemark[1] représentent les défauts qui compliquent parfois la vie de votre personnage. Certains sont plus contraignant que d'autres et certains dépendent du scénario. Ils permettent de choisir des Atouts supplémentaires mais surtout ajoute du caractère à votre personnage et augmente ainsi sont intérêt et votre plaisir de jouer.

\footnotetext[1]{Comme pour les compétences, je rentre pas dans le détail, hors mis pour les Handicaps qui changent par rapport à l'original de Savage World.}

\begin{description}[align=left]
    \item [Âgé (Majeur)]
        Votre héros se fait vieux, certes, mais n’est pas encore prêt pour l’hospice.
    \item [Anémique (Mineur)]
        Votre héros est particulièrement sensible aux maladies, à la fatigue et aux effets de l’environnement.
    \item [Arrogant (Majeur)]
        Votre héros ne pense pas qu’il est le meilleur, il sait qu’il l’est.
    \item [Aveugle (Majeur)]
        Votre héros complètement aveugle, il ne voit rien du tout.
    \item [Bavard (Mineur)]
        On dit qu’une grande gueule peut couler un navire. Votre héros l’est au point qu’il pourrait couler une armada.
    \item [Bizarrerie (Mineur)]
        Votre héros possède une bizarrerie, rien de grave, mais pouvant lui causer des ennuis. Genre ne pas se battre avant d'être présenter, frapper trois fois à la porte avant de pouvoir entrer\ldots
    \item [Boiteux (Majeur)]
        Une vieille blessure a presque estropié votre héros.
    \item [Borgne (Majeur)]
        Votre héros a dans le passé eu l’\oe{il} crevé par un ennemi.
    \item [Chimères (Mineur ou Majeur)]
        Votre héros croit en des choses qui paraissent étranges pour les autres.
    \item [Couard (Majeur)]
        Tout le monde n’a pas des nerfs d’acier. Votre héros blêmit à la vue du sang et est terrifié à l’idée même de subir une blessure.
    \item [Code d’Honneur (Majeur)]
        L’honneur est très important aux yeux de votre personnage.
    \item [Cupide (Mineur ou Majeur)]
        Votre héros est un grippe-sou qui mesure sa valeur en crédits.
    \item [Curieux (Majeur)]
        C’est un vilain défaut et votre personnage est très vilain.
    \item [Deux mains gauches (Mineur)]
        Certaines personnes ne sont tout simplement pas douées avec la technologie.
    \item [Dur d’Oreille (Mineur ou Majeur)]
        Un personnage ayant perdu tout ou partie de son acuité audition possède ce handicap.
    \item [Ennemi (Mineur ou Majeur)]
        Quelqu’un quelque part déteste votre héros et veut sa mort.
    \item [Étranger (Mineur)]
        Votre héros n’appartient pas à la société dans laquelle il vit.
    \item [Frêle (Majeur)]
        Votre héros est, soit très petit, soit très maigre, voire les deux par rapport à la norme de son peuple.
    \item [Gamin (Majeur)]
        Il arrive que des événements incroyables poussent des enfants vers l’aventure. Sachez que choisir ce Handicap signifie débuter avec un gros désavantage.    
    \item [Héroïque (Majeur)]
        Votre héros ne dit jamais non à une personne dans le besoin. Cela peut ne pas lui plaire mais il se sent obligé de secourir ceux qu’il considère sans défense.    
    \item [Ignorant (Majeur)]
        Votre héros en sait moins que les autres sur le monde dans lequel il vit.
    \item [Illettré (Mineur)]
        Votre héros ne sait pas lire.
    \item [Loyal (Mineur)]
        Votre personnage n’est peut-être pas l’archétype du héros, mais il donnerait sa vie pour ses amis.
    \item [Malchanceux (Majeur)]
        Votre héros est moins chanceux que les autres.
    \item [Manchot (Majeur)]
        Votre héros est né avec un seul bras ou l’a perdu lors d’un combat passé.
    \item [Mauvaise habitude (Mineur ou Majeur)]
        Votre héros possède une manie grossière et porte sur les nerfs de son entourage.
    \item [Moche (Mineur)]
        La nature n’a pas été tendre concernant l’apparence de votre héros.
    \item [Myope (Mineur ou Majeur)]
        La vue de votre héros n’est plus ce qu’elle était.
    \item [Obèse (Mineur)]
        Les personnes corpulentes éprouvent rapidement de grandes difficultés lors des situations physiques dangereuses.
    \item [Pacifiste (Mineur ou Majeur)]
        Votre héros déteste la violence et ne se bat jamais ou en d'exceptionnelles circonstences.
    \item [Phobie (Mineur ou Majeur)]
        Les phobies sont des peurs irrationnelles qui empoisonnent le héros pour le reste de sa vie.
    \item [Poches percées (Mineur)]
        Le dicton dit qu’un idiot et son argent ne font pas bon ménage. Votre héros est l’idiot en question.
    \item [Prudent (Mineur)]
        Ce personnage incarne l’extrême prudence. Il ne prend aucune décision hâtive.
    \item [Présomptueux (Majeur)]
        Rien ne peut résister à votre héros. C’est du moins ce qu’il croit.
    \item [Rancunier (Mineur ou Majeur)]
        Votre héros n’oublie jamais les offenses qui lui sont faites. La vengence est un plat qui se mange froid.
    \item [Recherché (Mineur ou Majeur)]
        Votre héros a commis un crime dans son passé et sera arrêté s’il est repris par les autorités.
    \item [Rien à perdre (Mineur)]
        N’avoir Rien à perdre ne signifie pas que votre héros est suicidaire mais qu’il ne tient à la vie que pour atteindre un but qu’il s’est fixé.
    \item [Sale caractère (Mineur)]
        Votre héros est désagréable et a très mauvais caractère.
    \item [Sanguinaire (Majeur)]
        Votre héros ne fait jamais de prisonniers à moins d’être sous la surveillance étroite d’un supérieur.
    \item [Sceptique (Mineur)]
        Les sceptiques ne croient au surnaturel que quand ils sont à moitié avalés par une créature surnaturelle.
    \item [Serment (Jedi ou Sith) (Majeur)]
        Le personnage a prêté serment il ne peut s'y soustraire.
    \item [Têtu (Mineur)]
        Votre héros veut toujours avoir raison et n’admettra jamais ses torts.
    \item [Unijambiste (Majeur)]
        Avec une béquille (ou une jambe de bois).
\end{description}

\begin{paperbox}{Serment (Jedi / Sith)}
    Cet handicap n'est accessible que pour personnages disposant de l'attribut \^Ame à d12 au moins et l'attribut Arcane (Force) enfin le personnage doit être au minimum Vétéran. 

    En effet cet handicap représente l'appartenant à un ordre, Jedi ou Sith et donne accès aux Atouts Padawan/Apprenti et Maître Jedi/Seigneur Sith. Avant de pouvoir préter serment, un héro doit maitriser la force et suivre une formation c'est pourquoi il ne peur accéder à cet Handicap en tant que Novice ou Aguerri.

    Cet handicap oblige le joueur à suivre le code de son ordre à la letter sous peine de succomber à la Force et de devenir incontrolable (PNJ).

    Voir le chapitre sur La Force~\ref{sec:force} (p. \pageref{sec:force}) pour plus de détails.

\end{paperbox}

\clearpage
\subsection{Atouts}

Voici le liste des Atouts disponibles dans \swfe.

\subsubsection{Background}
\begin{description}[align=left]
    \item [Ambidextre]
    	\emph{[Novice, Agi d8+]}\\
        Votre héros utilise ses deux mains avec la même facilité. Il ignore le malus de -2 habituel pour la main non-directrice.

    \item [Arcane (Force)]
    	\emph{[Novice, spécial, \^Ame d8+]}\\
        L'atout indispensable si votre héro doit utiliser la Force. Voir la section sur La Force~\ref{sec:force} (p. \pageref{sec:force}).

    \item [Brave]
    	\emph{[Novice, spécial]}\\
        Ceux qui possèdent cet Atout ont appris à maîtriser leurs peurs, à moins qu'ils ne soient tout simplement émotionnellement distants de tout. Quoi qu'il en soit, ils bénéficient d'un bonus de +2 sur les tests de terreur.

    \item [Chanceux]
    	\emph{[Novice]}\\
        Votre héros semble être béni par le destin. Il obtient un Jeton de plus au début de chaque session de jeu, ce qui lui octroie plus de chance de réussir ses actions et de survivre à d'incroyables dangers.

    \item [Très Chanceux]
    	\emph{[Novice, Chanceux]}\\
        Votre héros obtient 2 Jetons de plus au début de chaque session de jeu.

    \item [Costaud]
    	\emph{[Novice, Force et Vigueur d6+]}\\
        Votre héros est très corpulent ou tout simplement très athlétique. Dans tous les cas, sa masse lui permet de mieux résister aux dégâts. Ajoutez +1 à sa Résistance.

		En outre, votre héros est également capable de porter 4 fois sa Force en kg sans malus, au lieu de 2 fois chez les gens ordinaires.

    \item [Don des langues]
    	\emph{[Novice, Intellect d6+]}\\
        Le personnage a un don pour les langues : il commence le jeu en connaissant un nombre de langues égal à son Intellect, et peut faire un jet d'Intellect à -2 pour se faire comprendre en n'importe quel langage qu'il a côtoyé pendant au moins une semaine.

    \item [Enragé]
    	\emph{[Novice]}\\
        Dès qu'il est blessé (ou même Secoué par une attaque physique) votre héros doit réussir un jet d'Intellect sans quoi il devient Enragé. L'état Enragé inflige un malus de 2 à la Parade du personnage mais ajoute +2 à sa Résistance, à tous ses jets de Combat et de Force ainsi qu'aux dégâts causé au corps à corps. 

        Le héros ignore en outre tous les malus de blessures, mais il ne peut se servir de Compétences, d'Atout, ou effectuer des man\oe{uvres} qui requièrent de la concentration comme Tir et Sarcasme. L'utilisation de la Compétence Intimidation est toutefois toujours possible. Un personnage enragé attaque sans aucune retenue. S'il obtient un résultat de 1 sur son dé de Combat (sans tenir compte du dé Joker) il touche au hasard une cible adjacente autre que le cible prévue, ami ou ennemi. En l'absence de cible adjacente le coup est juste manqué. Un personnage enragé peut calmer sa rage s'il ne fait aucune action (ni même se déplacer) pendant un round entier et s'il réussit un jet d'Intellect à -2.

    \item [Guérison rapide]
    	\emph{[Novice, Vigueur d8+]}\\
        Votre héros récupère vite de ses blessures. Il obtient un bonus de +2 à ses jets de Vigueur pour guérison naturelle.

    \item [Noble]
    	\emph{[Novice]}\\
        Les personnages de haute naissance sont avantagés par la vie mais ont aussi plus de responsabilités. Un Noble possède un statut élevé dans sa société, a droit à un traitement de faveur de la part de ses ennemis, voit son Charisme augmenté de 2 et obtient l'Atout Riche. Il obtient plusieurs Atouts pour le prix d'un mais ses responsabilités en tant qu'aristocrate équilibrent ces avantages. 

        Un Noble dispose de troupes à ses ordres, de terres, d'une demeure ancestrale et d'autres biens à gérer. Ces aspects sont à définir par le MJ et doivent inclure la charge qui pèse sur le personnage. Un Noble disposera d'une troupe de soldats, d'une forteresse. Mais cela signifiera aussi qu'il devra gouverner sa province, rendre la justice et surveiller son voisin jaloux qui convoite ses terres, complotant sans cesse contre lui.

    \item [Résistance à la Force]
    	\emph{[Novice, \^Ame d8+]}\\
        Votre héro, même s'il ne sait pas l'utiliser, est capable de résister à la Force quand elle est employé contre lui. Il obtient 2 points d'Armure contre les dégâts provoqués par des pouvoirs de la Force et bénéficie d'un bonus de +2 à ses jets pour résister aux effets de ces pouvoirs.

    \item [Grande résistance à la Force]
    	\emph{[Novice, Résistance à la Force]}\\
        Idem que ci-dessus mais l'Armure et la résistance passent à 4.

    \item [Riche]
    	\emph{[Novice]}\\
        Que votre héros soit né avec une cuillère en argent dans la bouche ou bien qu'il ait bien réussi en affaires, une chose est sûre : il est beaucoup plus fortuné que la plupart des gens.

		Un héros Riche débute avec 3x les fonds initiaux prévus. Si le héro travaille il reçoit un salaire annuel de 150 000 Cr.

    \item [Très riche]
    	\emph{[Novice, Riche ou Noble]}\\
        Votre héros est riche comme Crésus. Il débute avec les 5x les fonds initiaux prévus, et, si c'est approprié, reçoit un revenu annuel de 500 000 Cr. Un héros Très riche doit avoir une histoire personnelle très détaillée à valider avec le MJ. Les biens et la fortune qu'il possède impliquent autant de gestion que de lourdes responsabilités.

    \item [Séduisant]
    	\emph{[Novice, Vigueur d6+]}\\
        Votre héros a beaucoup de charme ou est très beau. Son Charisme est augmenté de 2.

    \item [Très séduisant]
    	\emph{[Novice, Séduisant]}\\
        otre héros est d’une beauté à couper le souffle. Son Charisme est augmenté de 4.

    \item [Véloce]
    	\emph{[Novice, Agilité d6+]}\\
        Le héros se déplace très rapidement. Son Allure est augmentée de 2 et il utilise un d10 au lieu du d6 lorsqu'il court.

    \item [Vif]
    	\emph{[Novice, Agilité d8+]}\\
        Votre héros est né avec des réflexes presque surhumains. Si vous piochez un 5 ou une carte inférieure en combat, vous pouvez la défausser et en piochez une autre jusqu'à ce que vous obteniez une carte supérieure à 5. 

        Ceux qui ont les Atouts Tête froide et Sangfroid piochent d'abord toutes les cartes supplémentaires liées à ces Atouts, puis choisissent la meilleure de toutes avant d'utiliser l'Atout Vif.

    \item [Vigilant]
    	\emph{[Novice]}\\
        Peu de choses échappent à votre héros. Il est vigilant et très observateur. Il ajoute +2 à tous ses jets de Perception pour entendre, regarder ou percevoir le monde qui l'entoure.
\end{description}

\begin{paperbox}{Arcane (Force)}
    Remplace l'atout Arcanes. C'est l'atout qui représente la Force, l\^Ame étant assimilé au niveau de maîtrise de la Force.
\end{paperbox}

\begin{paperbox}{Résistance à la Force}
    Comme sont niveau supérieur Grande Résistance à la Force, cet atout remplace la résistance aux Arcanes. il correspond comme son nom l'indique à la capacité que le héro a de résisté à la force.

    Il confère au héro 2 points d'Armure contre les dégâts provoqués par la Force et ainsi qu'un bonus de +2 aux jets pour résister aux effets de ces pouvoirs. Utiliser un pouvoir bénéfique sur le personnage n'implique pas de malus, cet atout ne se déclenche qu'à la volonté du héro.

    Pour la grande résistance, le bonus d'Armure passe à 4.
\end{paperbox}

\subsubsection{Atouts de Combat}
\begin{description}[align=left]
    \item [Arme fétiche]
    	\emph{[Novice, Combat ou Tir d10+]}\\
        Le héros ne jure que par une arme qu'il connaît par c\oe{ur} (Arbalète Laser, Bâton électrique ou même ce bon vieux blaster Bryar). Quand il se bat avec cette arme il gagne +1 à ses jets de Tir ou de Combat. Il est possible de prendre cet Atout plusieurs fois pour des armes différentes. On peut remplacer une Arme fétiche perdue mais il faut patienter deux semaines pour que revienne le bonus de l'Atout.

    \item [Arme fétiche adorée]
    	\emph{[Vétéran, Arme fétiche]}\\
        Même chose que pour Arme fétiche mais le bonus est de +2.

    \item [Arts martiaux]
    	\emph{[Novice, Combat d6+]}\\
        Ce personnage est entrainé aux techniques de combat à mains nues et n'est du coup jamais sujet à la règle sur les Défenseurs désarmés. Lors d'une attaque réussie à mains nues, il fait des dégâts de For + d4, comme s'il portait une arme légère.

    \item [Maître des arts martiaux]
    	\emph{[Vétéran, Arts martiaux, Combat d10+]}\\
        Le personnage fait des dégâts de For + d6 lors d'une attaque à mains nues.

    \item [Bagarreur]
    	\emph{[Novice, Force d8+]}\\
        Votre héros a l'habitude de cogner avec ses poings, et fort ! Quand il touche une cible avec une attaque à mains nues, il ajoute +2 aux dégâts occasionnés.

    \item [Cogneur]
    	\emph{[Aguerri, Bagarreur]}\\
        Lorsqu'un cogneur obtient une Relance lors d'une attaque à mains nues, il rajoute un d8 au lieu d'un d6 aux dégâts.

    \item [Balayage]
    	\emph{[Novice, Force d8+, Combat d8+]}\\
        Cet Atout permet au personnage d'attaquer toutes les cibles adjacentes en un seul jet de Combat avec un malus de -2. Faites un jet de dégâts pour chaque cible touchée. Les alliés du héros sont affectés par cette attaque, aussi le héros devra rester prudent quant à l'utilisation de cet Atout. Il n'est pas possible d'utiliser les Atouts Balayage et Frénésie dans le même Round.

    \item [Grand balayage]
    	\emph{[Vétéran, Balayage]}\\
        Même chose que pour Balayage mais sans le malus de -2.

    \item [Blocage]
    	\emph{[Aguerri, Combat d8+]}\\
        Les héros endurcis au corps à corps sont plus habiles à se défendre que les autres. Ils savent attaquer mais ont aussi appris à bloquer les coups ennemis. Un héros qui possède cet Atout ajoute 1 à sa Parade.

    \item [Grand blocage]
    	\emph{[Vétéran, Blocage]}\\
        Même chose que pour Blocage mais le bonus est de +2.

    \item [Combat à deux armes]
    	\emph{[Novice, Agilité d8+]}\\
        Votre héros n'est pas ambidextre mais sait se battre avec deux armes en même temps. Quand il attaque avec une arme dans chaque main il fait un jet pour chaque attaque mais il ignore le -2 pour la pénalité d'Actions multiples.

    \item [Combatif]
    	\emph{[Aguerri]}\\
        Votre héros se ressaisit vite après un coup ou une émotion forte. Il ajoute +2 à ses jets d'\^Ame pour se remettre de l'état Secoué.

    \item [Contre-attaque]
    	\emph{[Aguerri, Combat d8+]}\\
        Les combattants disposant de cet Atout sont capables de répondre instantanément aux erreurs de leurs adversaires. Une fois par Round, lorsqu’un ennemi rate une attaque de corps à corps contre lui, le personnage peut faire une attaque gratuite en contre attaquant. L'attaque subit un malus de -2, et doit être une attaque normale (pas de désarmement, attaque totale ou toute autre man\oe{uvre}), et ne peut être combinée avec d'autres Atouts comme Frénésie ou Balayage. Il peut être utilisé avec la man\oe{uvre} Défense, mais pas avec Défense totale.

    \item [Grande contre-attaque]
    	\emph{[Vétéran, Contre-attaque]}\\
        Comme contre-attaque, mais sans le malus de -2.

    \item [Dégaine comme l'éclair]
    	\emph{[Novice, Agilité d8+]}\\
        Cet Atout permet au héros de dégainer une arme et d'attaquer dans le même round sans subir le malus usuel de -2. S'il doit faire un jet d'Agilité pour dégainer une arme il ajoute +2 à son jet.

    \item [Esquive]
    	\emph{[Aguerri, Agilité d8+]}\\
        L'expérience a appris à votre héros comment esquiver les mauvais coups. Grâce à cet Atout, il sait se mettre vite à couvert ou se déplacer rapidement pour éviter d'être touché. Ses ennemis ont un malus de -1 à leurs jets de Tir et de Lancer, sauf dans les cas d'une attaque surprise. Quand il tente d'échapper aux effets d'une attaque de zone il ajoute +1 à son jet d'Agilité (si la situation le permet).

    \item [Grande esquive]
    	\emph{[Vétéran, Esquive]}\\
        Même chose que pour Esquive mais les attaquants subissent -2 à leurs jets d’attaque et le héros ajoute +2 à son jet d’Agilité pour échapper aux effets d’une Attaque de zone.

    \item [Extraction]
    	\emph{[Novice, Agilité d8+]}\\
        Lorsqu'un personnage se retire d'un combat, les attaquants à son contact gagnent une attaque gratuite, une situation très risquée pour tous ! Fort heureusement, votre héros a l'habitude de ce genre de retraites.

		En réussissant un jet d'Agilité lors d'une retraite, un adversaire de votre choix ne disposera pas d'attaque gratuite.

    \item [Grande extraction]
    	\emph{[Novice, Extraction]}\\
        Comme Extraction, mais en cas de Relance sur le jet d'Agilité, aucun adversaire au contact du personnage ne bénéficiera d'une attaque gratuite.

    \item [Florentine]
    	\emph{[Novice, Agilité d8+, Combat d8+]}\\
        Combattre à la Florentine est une technique basée sur l'utilisation de deux armes à la fois. Le héros ajoute +1 à ses jets de Combat contre un adversaire muni d'une arme et sans bouclier. En outre, les bonus d'Attaque à plusieurs de ses ennemis sont diminués de 1 car le héros utilise avec brio ses deux armes pour parer leurs coups.

    \item [Frappe éclair]
    	\emph{[Novice, Agilité d8+]}\\
        Une fois par round votre héros obtient une attaque de Combat gratuite contre un seul ennemi venant au contact, à condition de ne pas être Secoué. Ceci interrompt automatiquement l'action de l'adversaire et ne coûte pas la sienne au héros s'il est en attente ou n'a pas encore fait d'action dans le round.

    \item [Frappe foudroyante]
    	\emph{[Héroïque, Frappe éclair]}\\
        Même chose que pour Frappe éclair, mais le héros obtient une attaque de Combat supplémentaire contre chacun des ennemis venant au contact.

    \item [Frénésie]
    	\emph{[Aguerri, Combat d10+]}\\
        Frénésie en combat de mêlée permet de faire pleuvoir des coups rapides sur ses adversaires au détriment de la précision. Le héros obtient une attaque de Combat en plus par round avec un malus de -2 à tous ses jets d'attaques. Cette attaque doit être appliquée en même temps qu'une autre attaque de Combat mais peut cibler deux ennemis différents adjacents au héros (les Joker lancent deux dés de Combat et un dé Joker). Le malus de -2 s'applique sur toutes les attaques. 

        Un personnage muni de deux armes ne fait toutefois qu'une attaque en plus.

    \item [Frénésie suprême]
    	\emph{[Vétéran, Frénésie]}\\
        Même chose que pour Frénésie mais sans le malus de -2.

    \item [Improvisation martiale]
    	\emph{[Aguerri, Intellect d6+]}\\
        Les héros se trouvent parfois avec du matériel non destiné au combat. Votre héros est capable d'utiliser ce genre d'objets en tant qu'armes improvisées. Il ne subit pas le malus de -1 normalement appliqué aux jets de Combat et à la Parade lorsqu'il s'en sert dans un combat au corps à corps.

    \item [Increvable]
    	\emph{[Joker, Novice, Âme d8+]}\\
        Votre héros a plus de vies qu'un camion rempli de chats. Lorsqu'il doit lancer les dés quand il passe dans un État critique ou sur la Table des blessures il ignore les malus dus aux blessures. Seul les jets de Vigueur sur ces tables sont concernés : il subit toujours les malus pour blessures pour les autres jets de Traits.

    \item [Trompe-la-mort]
    	\emph{[Joker, Novice, Âme d8+]}\\
        Votre héros est plus dur à tuer que Raspoutine lui-même. Si jamais il est tué, lancez un dé. Sur un résultat impair il est vraiment mort. Par contre, un résultat pair signifie que d'une façon ou d'une autre il échappe à la mort et n'est que dans un État critique. Il peut être fait prisonnier, être dépouillé ou simplement laissé pour mort, mais d'une manière ou d'une autre, il est toujours en vie.

    \item [Instinct de tueur]
    	\emph{[Héroïque]}\\
        Votre héros déteste perdre. Dans un cas d'égalité sur un jet opposé, il gagne. En outre, s'il fait un 1 sur son dé lors d'un jet Opposé, il peut relancer le dé (mais doit garder le second résultat, même s'il s'agit d'un nouveau 1).

    \item [Nerfs d’acier trempés]
    	\emph{[Novice, Nerfs d’Acier]}\\
        Même chose que pour Nerfs d'acier, mais 2 points de malus pour blessures sont ignorés.

    \item [Panache]
    	\emph{[Novice, \^Ame d8+]}\\
        Quand ce héros met tout son c\oe{ur} dans une tache, ça se voit ! Quand il dépense un Jeton sur un jet de Trait (y compris un jet d'encaissement), il ajoute +2 au résultat final.

    \item [Poigne ferme]
    	\emph{[Novice, Agilité d8+]}\\
        Votre héros ignore le malus de -2 de Plateforme instable pour tirer depuis un véhicule ou sur une monture en mouvement.

    \item [Rock'n'roll !]
    	\emph{[Aguerri, Tir d8+]}\\
        Les tireurs expérimentés ont appris à maîtriser le recul provoqué par les armes automatiques. Si un héros muni de cet Atout ne se déplace pas, il ignore le malus de -2 du à l'effet de recul de ce type d'armes.

    \item [Sans pitié]
    	\emph{[Aguerri]}\\
        Le personnage peut utiliser un Jeton pour relancer un jet de dégâts, y compris pour une attaque de zone.

    \item [Tête froide]
    	\emph{[Aguerri, Intellect d8+]}\\
        Celui qui garde son calme quand les autres courent aux abris est un adversaire redoutable. Cet Atout permet au héros de piocher une carte en plus pour l'Initiative. Il fait son action sur la meilleure des cartes.

    \item [Sang-froid]
    	\emph{[Aguerri, Tête froide]}\\
        Même chose que pour Tête froide, mais le héros pioche 2 cartes supplémentaires.

    \item [Tireur d’élite]
    	\emph{[Aguerri]}\\
        Le héros vise juste et bien. S'il ne bouge pas durant le round il tire comme s'il avait pris la man\oe{uvre} Viser. Cet Atout ne fonctionne pas avec une arme ayant une Cadence de Tir supérieure à 1.
		
		Cet Atout s'applique aux Compétences Tir et Lancer.

    \item [Tueur de géant]
    	\emph{[Vétéran]}\\
        En général plus c'est grand plus c'est dur à vaincre. Mais votre héros sait trouver les points faibles des créatures de grande taille. Votre héros ajoute 1d6 à ses dégâts lorsqu'il attaque des créatures d'au moins trois tailles supérieure à la sienne.

\newpage

    \item [Maître Jedi]
    	\emph{[Vétéran, \^Ame d12+, Padawan]}\\
        Après sa formation et après avoir prété serment à l'ordre, votre héro pourra devenir un Maître Jedi.

    \item [Seigneur Sith]
    	\emph{[Vétéran, \^Ame d12+, Apprenti Sith]}\\
        Votre héro, après avoir prété serment à son Maître, devient à son tour un seigneur Sith.

    \item [Padawan]
    	\emph{[Aguerri, \^Ame d10+, Arcane (Force)]}\\
        La sensibilité de votre héro à la force a été remarqué et l'ordre vous accepte comme padawan sous la tutelle d'un Maître.

    \item [Apprenti Sith]
    	\emph{[Aguerri, \^Ame d10+, Arcane (Force)]}\\
        La haine et la rancoeur de votre héro à donné des idées à un Seigneur Sith qui decide de vous prendre comme apprenti.


\end{description}

\subsubsection{Atouts de commandement}

\begin{description}[align=left]
    \item [Ambidextre]
    	\emph{[Novice, Agilité d8+]}\\
        Votre héros utilise ses deux mains avec la même facilité.

    \item [Ambidextre]
    	\emph{[Novice, Agilité d8+]}\\
        Votre héros utilise ses deux mains avec la même facilité.

    \item [Ambidextre]
    	\emph{[Novice, Agilité d8+]}\\
        Votre héros utilise ses deux mains avec la même facilité.

    \item [Ambidextre]
    	\emph{[Novice, Agilité d8+]}\\
        Votre héros utilise ses deux mains avec la même facilité.

    \item [Ambidextre]
    	\emph{[Novice, Agilité d8+]}\\
        Votre héros utilise ses deux mains avec la même facilité.

    \item [Ambidextre]
    	\emph{[Novice, Agilité d8+]}\\
        Votre héros utilise ses deux mains avec la même facilité.

    \item [Ambidextre]
    	\emph{[Novice, Agilité d8+]}\\
        Votre héros utilise ses deux mains avec la même facilité.

    \item [Ambidextre]
    	\emph{[Novice, Agilité d8+]}\\
        Votre héros utilise ses deux mains avec la même facilité.

    \item [Ambidextre]
    	\emph{[Novice, Agilité d8+]}\\
        Votre héros utilise ses deux mains avec la même facilité.


\end{description}

\newpage
\clearpage
\subsection{Création d'un personnage}

\subsubsection{Race}
Choisissez la race de votre personnage parmi celles proposées.

\subsubsection{Traits}
\begin{rebelist}
	\item Votre héros débute avec un d4 dans chaque Attribut et dispose de 5 points avec lesquels il peut les augmenter. Augmenter un Attribut d’un type de 	dé coûte 1 point.
	\item Vous avez 15 points pour vos Compétences.
	\item Chaque type de dé investi dans une Compétence coûte 1 point jusqu’à égaler le dé de l’Attribut dont dépend la Compétence. Au-delà, le coût est de 	2 points.
	\item Le Charisme est égal au total des bonus ou des malus donnés par les Atouts ou les Handicaps.
	\item L’Allure est de 6 cases.
	\item La Parade est égale à 2 plus la moitié de la Compétence Combat.
	\item La Résistance est égale à 2 plus la moitié de Vigueur. Vous pouvez y ajouter le bonus de l’armure portée sur le torse pour accélérer le calcul en 	cours de partie mais ce bonus ne compte pas pour les autres parties du corps.
\end{rebelist}

\subsubsection{Atouts \& Handicaps}
Vous gagnez des points supplémentaires en prenant un Handicap Majeur (2 points) et jusqu’à deux Handicaps Mineurs (1 point chacun).

Pour 2 points vous pouvez :
\begin{rebelist}
	\item Augmenter un Attribut d’un type de dé.
	\item Choisir un Atout.
\end{rebelist}

Pour 1 point vous pouvez :
\begin{rebelist}
	\item Augmenter une Compétence d’un type de dé.
	\item Doubler vos fonds initiaux.
\end{rebelist}

\subsubsection{Équipement}
Débutez avec 500\crg.

\subsubsection{Histoire personnelle}
Rajoutez les détails concernant le personnage qui vous semblent importants.

\subsection{Progression}
Le rang détermine la puissance approximative de votre héros. Le nombre de points d’XP que vous avez gagné détermine le rang selon le tableau suivant : 

\begin{itemtable}[ l l ]
	\textbf{XP}		& \textbf{Rang} \\
	00-19 			& Novice \\
   	20-39 			& Aguerri \\
   	40-59 			& Vétéran \\
   	60-79 			& Héroïque \\
   	80+				& Légendaire
\end{itemtable}

Tout les 5XP votre héros bénéficie d’une progression. Lors d’une progression il peut :
\begin{rebelist}
	\item Choisir un nouvel Atout.
	\item Augmenter une Compétence dont la valeur est égale ou supérieure au Trait associé.
	\item Augmenter deux Compétences dont les valeurs sont inférieures aux Traits associés.
	\item Prendre une nouvelle Compétence à d4.
	\item Augmenter un Attribut d’un type de dé. Vous ne pouvez choisir cette option qu’une fois par Rang ou lors d’une progression sur deux après avoir atteint le rang Légendaire. Il n’est pas possible de progresser au-delà de d12 dans un Trait par le biais d’une Progression, mais vous pouvez consulter les Atouts Professionnel~\ref{sec:atout-professionnel} et Expert~\ref{sec:atout-expert} (p.~\pageref{sec:atout-professionnel}).
\end{rebelist}

\subsection{Fiche de personnage}
Pour finir avec la création de personnage, je vous propose cette fiche de perso, initialement pour \citetitle{savage-worlds} mais qui est tout à fait utilisable puisqu’on a collé au plus à l’ouvrage d’origine.

\cite{torgan-savage-perso}

\clearpage 

