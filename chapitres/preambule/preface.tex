\onecolumn
\section{Préface}

\lettrine{P}{ourquoi} encore une autre adaptation de \citetitle{savage-worlds} à {\jedifont Star Wars} ? C’est vrai il en existe déjà plusieurs comme \citetitle{starwars-reloaded} ou \citetitle{starwars-unchained}. Mais toutes les versions que j’ai trouvée présentent des défauts: fautes d’orthographe ou de typo, traductions relatives, trop ou pas assez de détails, \ldots Mais surtout, aucune des versions que j’ai trouvées n’est collaborative, il n’ai pas possible de proposer des corrections ou des améliorations ni de se forker sa propre version.

C’est pour cela que j’ai eu envie de faire une nouvelle version, une version disponible sur Github que n’importe qui pourra améliorer, corriger ou forker à volonté. C’est comme ça qu’est né {\jedifont \doctitle}. Un MJ y trouvera tout ce dont ses joueurs ont besoin de savoir pour créer des fiches de perso et démarrer une partie. Bien sûr je n’ai pas retranscrit tout le bouquin original de Savage Worlds, mais les règles de base sont renseignées dans ce manuel afin que les joueurs n’ai pas à lire tout le bouquin de base. Le MJ lui n’y échappera pas. J’ai essayé autant que possible de rester dans l’esprit de Savage Worlds, simple et rapide à jouer et, au maximum, de réutiliser ce que le bouquin fournit.

De même, j’ai choisi de ne pas trop approfondir la description de l’univers de Star Wars et de me concentrer sur les adaptations de \citetitle{savage-worlds}. L’univers est largement assez connu et détaillé dans les sites spécialisés sur internet (\citetitle{website:starwars-holonet}, \citetitle{website:starwars-universe}, \ldots) et chaque MJ prendra les libertés qu’il souhaite dans son background. Ce manuel est à considérer comme le manuel du joueur, le MJ aura besoin du livre de base \citetitle{savage-worlds} pour mener la partie. Les informations présentes dans ce manuel ne suffisent pas aux MJs.

\subsection{Grosbilisme}
Dans l’univers Star Wars il est vite tentant de vouloir se faire un personnage Jedi. Mais les Jedi ne sont pas monnaie courante dans l’univers. De plus un Jedi n’aura de réelles difficultés que contre des Sith, l’intérêt d’un tel personnage n’est qu’illusoire, il limite le nombre d’ennemis intéressant et l’évolution de votre personnage. C’est pourquoi dans ce manuel, faire un Jedi est faisable mais ça se mérite, historique, évolution, \ldots Tout ça sera nécessaire pour faire un bon Jedi (ou un mauvais Sith).

\subsection{Contribution}
Comme je le disais plus haut, l’idée principale de ce document est de permettre à tous ceux qui le souhaitent de participer. Le document est donc écrit en \LaTeX et disponible sur Github.\\
\cite{jdrp-starwars}.

\subsection{Licence}
Cette \oe{uvre} est mise à disposition sous licence Attribution-ShareAlike 4.0 International. Pour voir une copie de cette licence, visitez \url{https://creativecommons.org/licenses/by-sa/4.0/} ou écrivez à Creative Commons, PO Box 1866, Mountain View, CA 94042, USA.

\begin{flushright}
	\includegraphics[width=70pt]{img/cc-by-sa.png}
\end{flushright}

\twocolumn